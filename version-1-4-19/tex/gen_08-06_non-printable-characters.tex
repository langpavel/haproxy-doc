% This is generated content
% Section 8.6.

\section{Non-printable characters}

\begin{verbatim}

In order not to cause trouble to log analysis tools or terminals during log
consulting, non-printable characters are not sent as-is into log files, but are
converted to the two-digits hexadecimal representation of their ASCII code,
prefixed by the character '#'. The only characters that can be logged without
being escaped are comprised between 32 and 126 (inclusive). Obviously, the
escape character '#' itself is also encoded to avoid any ambiguity ("#23"). It
is the same for the character '"' which becomes "#22", as well as '{', '|' and
'}' when logging headers.

Note that the space character (' ') is not encoded in headers, which can cause
issues for tools relying on space count to locate fields. A typical header
containing spaces is "User-Agent".

Last, it has been observed that some syslog daemons such as syslog-ng escape
the quote ('"') with a backslash ('\'). The reverse operation can safely be
performed since no quote may appear anywhere else in the logs.


\end{verbatim}
