% This is generated content
% Section 9.2.

\section{Unix Socket commands}

\begin{verbatim}

The following commands are supported on the UNIX stats socket ; all of them
must be terminated by a line feed. The socket supports pipelining, so that it
is possible to chain multiple commands at once provided they are delimited by
a semi-colon or a line feed, although the former is more reliable as it has no
risk of being truncated over the network. The responses themselves will each be
followed by an empty line, so it will be easy for an external script to match a
given response with a given request. By default one command line is processed
then the connection closes, but there is an interactive allowing multiple lines
to be issued one at a time.

It is important to understand that when multiple haproxy processes are started
on the same sockets, any process may pick up the request and will output its
own stats.

clear counters
  Clear the max values of the statistics counters in each proxy (frontend &
  backend) and in each server. The cumulated counters are not affected. This
  can be used to get clean counters after an incident, without having to
  restart nor to clear traffic counters. This command is restricted and can
  only be issued on sockets configured for levels "operator" or "admin".

clear counters all
  Clear all statistics counters in each proxy (frontend & backend) and in each
  server. This has the same effect as restarting. This command is restricted
  and can only be issued on sockets configured for level "admin".

disable server <backend>/<server>
  Mark the server DOWN for maintenance. In this mode, no more checks will be
  performed on the server until it leaves maintenance.
  If the server is tracked by other servers, those servers will be set to DOWN
  during the maintenance.

  In the statistics page, a server DOWN for maintenance will appear with a
  "MAINT" status, its tracking servers with the "MAINT(via)" one.

  Both the backend and the server may be specified either by their name or by
  their numeric ID, prefixed with a sharp ('#').

  This command is restricted and can only be issued on sockets configured for
  level "admin".

enable server <backend>/<server>
  If the server was previously marked as DOWN for maintenance, this marks the
  server UP and checks are re-enabled.

  Both the backend and the server may be specified either by their name or by
  their numeric ID, prefixed with a sharp ('#').

  This command is restricted and can only be issued on sockets configured for
  level "admin".

get weight <backend>/<server>
  Report the current weight and the initial weight of server <server> in
  backend <backend> or an error if either doesn't exist. The initial weight is
  the one that appears in the configuration file. Both are normally equal
  unless the current weight has been changed. Both the backend and the server
  may be specified either by their name or by their numeric ID, prefixed with a
  sharp ('#').

help
  Print the list of known keywords and their basic usage. The same help screen
  is also displayed for unknown commands.

prompt
  Toggle the prompt at the beginning of the line and enter or leave interactive
  mode. In interactive mode, the connection is not closed after a command
  completes. Instead, the prompt will appear again, indicating the user that
  the interpreter is waiting for a new command. The prompt consists in a right
  angle bracket followed by a space "> ". This mode is particularly convenient
  when one wants to periodically check information such as stats or errors.
  It is also a good idea to enter interactive mode before issuing a "help"
  command.

quit
  Close the connection when in interactive mode.

set timeout cli <delay>
  Change the CLI interface timeout for current connection. This can be useful
  during long debugging sessions where the user needs to constantly inspect
  some indicators without being disconnected. The delay is passed in seconds.

set weight <backend>/<server> <weight>[%]
  Change a server's weight to the value passed in argument. If the value ends
  with the '%' sign, then the new weight will be relative to the initially
  configured weight. Relative weights are only permitted between 0 and 100%,
  and absolute weights are permitted between 0 and 256. Servers which are part
  of a farm running a static load-balancing algorithm have stricter limitations
  because the weight cannot change once set. Thus for these servers, the only
  accepted values are 0 and 100% (or 0 and the initial weight). Changes take
  effect immediately, though certain LB algorithms require a certain amount of
  requests to consider changes. A typical usage of this command is to disable
  a server during an update by setting its weight to zero, then to enable it
  again after the update by setting it back to 100%. This command is restricted
  and can only be issued on sockets configured for level "admin". Both the
  backend and the server may be specified either by their name or by their
  numeric ID, prefixed with a sharp ('#').

show errors [<iid>]
  Dump last known request and response errors collected by frontends and
  backends. If <iid> is specified, the limit the dump to errors concerning
  either frontend or backend whose ID is <iid>. This command is restricted
  and can only be issued on sockets configured for levels "operator" or
  "admin".

  The errors which may be collected are the last request and response errors
  caused by protocol violations, often due to invalid characters in header
  names. The report precisely indicates what exact character violated the
  protocol. Other important information such as the exact date the error was
  detected, frontend and backend names, the server name (when known), the
  internal session ID and the source address which has initiated the session
  are reported too.

  All characters are returned, and non-printable characters are encoded. The
  most common ones (\t = 9, \n = 10, \r = 13 and \e = 27) are encoded as one
  letter following a backslash. The backslash itself is encoded as '\\' to
  avoid confusion. Other non-printable characters are encoded '\xNN' where
  NN is the two-digits hexadecimal representation of the character's ASCII
  code.

  Lines are prefixed with the position of their first character, starting at 0
  for the beginning of the buffer. At most one input line is printed per line,
  and large lines will be broken into multiple consecutive output lines so that
  the output never goes beyond 79 characters wide. It is easy to detect if a
  line was broken, because it will not end with '\n' and the next line's offset
  will be followed by a '+' sign, indicating it is a continuation of previous
  line.

  Example :
    >>> $ echo "show errors" | socat stdio /tmp/sock1
        [04/Mar/2009:15:46:56.081] backend http-in (#2) : invalid response
          src 127.0.0.1, session #54, frontend fe-eth0 (#1), server s2 (#1)
          response length 213 bytes, error at position 23:

          00000  HTTP/1.0 200 OK\r\n
          00017  header/bizarre:blah\r\n
          00038  Location: blah\r\n
          00054  Long-line: this is a very long line which should b
          00104+ e broken into multiple lines on the output buffer,
          00154+  otherwise it would be too large to print in a ter
          00204+ minal\r\n
          00211  \r\n

    In the example above, we see that the backend "http-in" which has internal
    ID 2 has blocked an invalid response from its server s2 which has internal
    ID 1. The request was on session 54 initiated by source 127.0.0.1 and
    received by frontend fe-eth0 whose ID is 1. The total response length was
    213 bytes when the error was detected, and the error was at byte 23. This
    is the slash ('/') in header name "header/bizarre", which is not a valid
    HTTP character for a header name.

show info
  Dump info about haproxy status on current process.

show sess
  Dump all known sessions. Avoid doing this on slow connections as this can
  be huge. This command is restricted and can only be issued on sockets
  configured for levels "operator" or "admin".

show sess <id>
  Display a lot of internal information about the specified session identifier.
  This identifier is the first field at the beginning of the lines in the dumps
  of "show sess" (it corresponds to the session pointer). Those information are
  useless to most users but may be used by haproxy developers to troubleshoot a
  complex bug. The output format is intentionally not documented so that it can
  freely evolve depending on demands.

show stat [<iid> <type> <sid>]
  Dump statistics in the CSV format. By passing <id>, <type> and <sid>, it is
  possible to dump only selected items :
    - <iid> is a proxy ID, -1 to dump everything
    - <type> selects the type of dumpable objects : 1 for frontends, 2 for
       backends, 4 for servers, -1 for everything. These values can be ORed,
       for example:
          1 + 2     = 3   -> frontend + backend.
          1 + 2 + 4 = 7   -> frontend + backend + server.
    - <sid> is a server ID, -1 to dump everything from the selected proxy.

  Example :
    >>> $ echo "show info;show stat" | socat stdio unix-connect:/tmp/sock1
        Name: HAProxy
        Version: 1.4-dev2-49
        Release_date: 2009/09/23
        Nbproc: 1
        Process_num: 1
        (...)

        # pxname,svname,qcur,qmax,scur,smax,slim,stot,bin,bout,dreq,  (...)
        stats,FRONTEND,,,0,0,1000,0,0,0,0,0,0,,,,,OPEN,,,,,,,,,1,1,0, (...)
        stats,BACKEND,0,0,0,0,1000,0,0,0,0,0,,0,0,0,0,UP,0,0,0,,0,250,(...)
        (...)
        www1,BACKEND,0,0,0,0,1000,0,0,0,0,0,,0,0,0,0,UP,1,1,0,,0,250, (...)

        $

    Here, two commands have been issued at once. That way it's easy to find
    which process the stats apply to in multi-process mode. Notice the empty
    line after the information output which marks the end of the first block.
    A similar empty line appears at the end of the second block (stats) so that
    the reader knows the output has not been truncated.


/*
 * Local variables:
 *  fill-column: 79
 * End:
 */

\end{verbatim}
