% This is generated content
% Section 7.5.3.

\subsection{Matching at Layer 7}

\begin{verbatim}

A third set of criteria applies to information which can be found at the
application layer (layer 7). Those require that a full HTTP request has been
read, and are only evaluated then. They may require slightly more CPU resources
than the layer 4 ones, but not much since the request and response are indexed.

hdr <string>
hdr(header) <string>
  Note: all the "hdr*" matching criteria either apply to all headers, or to a
  particular header whose name is passed between parenthesis and without any
  space. The header name is not case-sensitive. The header matching complies
  with RFC2616, and treats as separate headers all values delimited by commas.
  Use the shdr() variant for response headers sent by the server.

  The "hdr" criteria returns true if any of the headers matching the criteria
  match any of the strings. This can be used to check exact for values. For
  instance, checking that "connection: close" is set :

     hdr(Connection) -i close

hdr_beg <string>
hdr_beg(header) <string>
  Returns true when one of the headers begins with one of the strings. See
  "hdr" for more information on header matching. Use the shdr_beg() variant for
  response headers sent by the server.

hdr_cnt <integer>
hdr_cnt(header) <integer>
  Returns true when the number of occurrence of the specified header matches
  the values or ranges specified. It is important to remember that one header
  line may count as several headers if it has several values. This is used to
  detect presence, absence or abuse of a specific header, as well as to block
  request smuggling attacks by rejecting requests which contain more than one
  of certain headers. See "hdr" for more information on header matching. Use
  the shdr_cnt() variant for response headers sent by the server.

hdr_dir <string>
hdr_dir(header) <string>
  Returns true when one of the headers contains one of the strings either
  isolated or delimited by slashes. This is used to perform filename or
  directory name matching, and may be used with Referer. See "hdr" for more
  information on header matching. Use the shdr_dir() variant for response
  headers sent by the server.

hdr_dom <string>
hdr_dom(header) <string>
  Returns true when one of the headers contains one of the strings either
  isolated or delimited by dots. This is used to perform domain name matching,
  and may be used with the Host header. See "hdr" for more information on
  header matching. Use the shdr_dom() variant for response headers sent by the
  server.

hdr_end <string>
hdr_end(header) <string>
  Returns true when one of the headers ends with one of the strings. See "hdr"
  for more information on header matching. Use the shdr_end() variant for
  response headers sent by the server.

hdr_ip <ip_address>
hdr_ip(header) <ip_address>
  Returns true when one of the headers' values contains an IP address matching
  <ip_address>. This is mainly used with headers such as X-Forwarded-For or
  X-Client-IP. See "hdr" for more information on header matching. Use the
  shdr_ip() variant for response headers sent by the server.

hdr_len <integer>
hdr_len(<header>) <integer>
  Returns true when at least one of the headers has a length which matches the
  values or ranges specified. This may be used to detect empty or too large
  headers. See "hdr" for more information on header matching. Use the
  shdr_len() variant for response headers sent by the server.

hdr_reg <regex>
hdr_reg(header) <regex>
  Returns true when one of the headers matches of the regular expressions. It
  can be used at any time, but it is important to remember that regex matching
  is slower than other methods. See also other "hdr_" criteria, as well as
  "hdr" for more information on header matching. Use the shdr_reg() variant for
  response headers sent by the server.

hdr_sub <string>
hdr_sub(header) <string>
  Returns true when one of the headers contains one of the strings. See "hdr"
  for more information on header matching. Use the shdr_sub() variant for
  response headers sent by the server.

hdr_val <integer>
hdr_val(header) <integer>
  Returns true when one of the headers starts with a number which matches the
  values or ranges specified. This may be used to limit content-length to
  acceptable values for example. See "hdr" for more information on header
  matching. Use the shdr_val() variant for response headers sent by the server.

http_auth(userlist)
http_auth_group(userlist) <group> [<group>]*
  Returns true when authentication data received from the client matches
  username & password stored on the userlist. It is also possible to
  use http_auth_group to check if the user is assigned to at least one
  of specified groups.

  Currently only http basic auth is supported.

http_first_req
  Returns true when the request being processed is the first one of the
  connection. This can be used to add or remove headers that may be missing
  from some requests when a request is not the first one, or even to perform
  some specific ACL checks only on the first request.

method <string>
  Applies to the method in the HTTP request, eg: "GET". Some predefined ACL
  already check for most common methods.

path <string>
  Returns true when the path part of the request, which starts at the first
  slash and ends before the question mark, equals one of the strings. It may be
  used to match known files, such as /favicon.ico.

path_beg <string>
  Returns true when the path begins with one of the strings. This can be used
  to send certain directory names to alternative backends.

path_dir <string>
  Returns true when one of the strings is found isolated or delimited with
  slashes in the path. This is used to perform filename or directory name
  matching without the risk of wrong match due to colliding prefixes. See also
  "url_dir" and "path_sub".

path_dom <string>
  Returns true when one of the strings is found isolated or delimited with dots
  in the path. This may be used to perform domain name matching in proxy
  requests. See also "path_sub" and "url_dom".

path_end <string>
  Returns true when the path ends with one of the strings. This may be used to
  control file name extension.

path_len <integer>
  Returns true when the path length matches the values or ranges specified.
  This may be used to detect abusive requests for instance.

path_reg <regex>
  Returns true when the path matches one of the regular expressions. It can be
  used any time, but it is important to remember that regex matching is slower
  than other methods. See also "url_reg" and all "path_" criteria.

path_sub <string>
  Returns true when the path contains one of the strings. It can be used to
  detect particular patterns in paths, such as "../" for example. See also
  "path_dir".

req_ver <string>
  Applies to the version string in the HTTP request, eg: "1.0". Some predefined
  ACL already check for versions 1.0 and 1.1.

status <integer>
  Applies to the HTTP status code in the HTTP response, eg: "302". It can be
  used to act on responses depending on status ranges, for instance, remove
  any Location header if the response is not a 3xx.

url <string>
  Applies to the whole URL passed in the request. The only real use is to match
  "*", for which there already is a predefined ACL.

url_beg <string>
  Returns true when the URL begins with one of the strings. This can be used to
  check whether a URL begins with a slash or with a protocol scheme.

url_dir <string>
  Returns true when one of the strings is found isolated or delimited with
  slashes in the URL. This is used to perform filename or directory name
  matching without the risk of wrong match due to colliding prefixes. See also
  "path_dir" and "url_sub".

url_dom <string>
  Returns true when one of the strings is found isolated or delimited with dots
  in the URL. This is used to perform domain name matching without the risk of
  wrong match due to colliding prefixes. See also "url_sub".

url_end <string>
  Returns true when the URL ends with one of the strings. It has very limited
  use. "path_end" should be used instead for filename matching.

url_ip <ip_address>
  Applies to the IP address specified in the absolute URI in an HTTP request.
  It can be used to prevent access to certain resources such as local network.
  It is useful with option "http_proxy".

url_len <integer>
  Returns true when the url length matches the values or ranges specified. This
  may be used to detect abusive requests for instance.

url_port <integer>
  Applies to the port specified in the absolute URI in an HTTP request. It can
  be used to prevent access to certain resources. It is useful with option
  "http_proxy". Note that if the port is not specified in the request, port 80
  is assumed.

url_reg <regex>
  Returns true when the URL matches one of the regular expressions. It can be
  used any time, but it is important to remember that regex matching is slower
  than other methods. See also "path_reg" and all "url_" criteria.

url_sub <string>
  Returns true when the URL contains one of the strings. It can be used to
  detect particular patterns in query strings for example. See also "path_sub".


\end{verbatim}
