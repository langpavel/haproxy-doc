% This is generated content
% Section 1.2.1.

\subsection{The Request line}

\begin{verbatim}

Line 1 is the "request line". It is always composed of 3 fields :

  - a METHOD      : GET
  - a URI         : /serv/login.php?lang=en&profile=2
  - a version tag : HTTP/1.1

All of them are delimited by what the standard calls LWS (linear white spaces),
which are commonly spaces, but can also be tabs or line feeds/carriage returns
followed by spaces/tabs. The method itself cannot contain any colon (':') and
is limited to alphabetic letters. All those various combinations make it
desirable that HAProxy performs the splitting itself rather than leaving it to
the user to write a complex or inaccurate regular expression.

The URI itself can have several forms :

  - A "relative URI" :

      /serv/login.php?lang=en&profile=2

    It is a complete URL without the host part. This is generally what is
    received by servers, reverse proxies and transparent proxies.

  - An "absolute URI", also called a "URL" :

      http://192.168.0.12:8080/serv/login.php?lang=en&profile=2

    It is composed of a "scheme" (the protocol name followed by '://'), a host
    name or address, optionally a colon (':') followed by a port number, then
    a relative URI beginning at the first slash ('/') after the address part.
    This is generally what proxies receive, but a server supporting HTTP/1.1
    must accept this form too.

  - a star ('*') : this form is only accepted in association with the OPTIONS
    method and is not relayable. It is used to inquiry a next hop's
    capabilities.

  - an address:port combination : 192.168.0.12:80
    This is used with the CONNECT method, which is used to establish TCP
    tunnels through HTTP proxies, generally for HTTPS, but sometimes for
    other protocols too.

In a relative URI, two sub-parts are identified. The part before the question
mark is called the "path". It is typically the relative path to static objects
on the server. The part after the question mark is called the "query string".
It is mostly used with GET requests sent to dynamic scripts and is very
specific to the language, framework or application in use.


\end{verbatim}
