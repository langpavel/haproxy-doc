% This is generated content
% Section 8.2.2.

\subsection{TCP log format}

\begin{verbatim}

The TCP format is used when "option tcplog" is specified in the frontend, and
is the recommended format for pure TCP proxies. It provides a lot of precious
information for troubleshooting. Since this format includes timers and byte
counts, the log is normally emitted at the end of the session. It can be
emitted earlier if "option logasap" is specified, which makes sense in most
environments with long sessions such as remote terminals. Sessions which match
the "monitor" rules are never logged. It is also possible not to emit logs for
sessions for which no data were exchanged between the client and the server, by
specifying "option dontlognull" in the frontend. Successful connections will
not be logged if "option dontlog-normal" is specified in the frontend. A few
fields may slightly vary depending on some configuration options, those are
marked with a star ('*') after the field name below.

  Example :
        frontend fnt
            mode tcp
            option tcplog
            log global
            default_backend bck

        backend bck
            server srv1 127.0.0.1:8000

    >>> Feb  6 12:12:56 localhost \
          haproxy[14387]: 10.0.1.2:33313 [06/Feb/2009:12:12:51.443] fnt \
          bck/srv1 0/0/5007 212 -- 0/0/0/0/3 0/0

  Field   Format                                Extract from the example above
      1   process_name '[' pid ']:'                            haproxy[14387]:
      2   client_ip ':' client_port                             10.0.1.2:33313
      3   '[' accept_date ']'                       [06/Feb/2009:12:12:51.443]
      4   frontend_name                                                    fnt
      5   backend_name '/' server_name                                bck/srv1
      6   Tw '/' Tc '/' Tt*                                           0/0/5007
      7   bytes_read*                                                      212
      8   termination_state                                                 --
      9   actconn '/' feconn '/' beconn '/' srv_conn '/' retries*    0/0/0/0/3
     10   srv_queue '/' backend_queue                                      0/0

Detailed fields description :
  - "client_ip" is the IP address of the client which initiated the TCP
    connection to haproxy.

  - "client_port" is the TCP port of the client which initiated the connection.

  - "accept_date" is the exact date when the connection was received by haproxy
    (which might be very slightly different from the date observed on the
    network if there was some queuing in the system's backlog). This is usually
    the same date which may appear in any upstream firewall's log.

  - "frontend_name" is the name of the frontend (or listener) which received
    and processed the connection.

  - "backend_name" is the name of the backend (or listener) which was selected
    to manage the connection to the server. This will be the same as the
    frontend if no switching rule has been applied, which is common for TCP
    applications.

  - "server_name" is the name of the last server to which the connection was
    sent, which might differ from the first one if there were connection errors
    and a redispatch occurred. Note that this server belongs to the backend
    which processed the request. If the connection was aborted before reaching
    a server, "<NOSRV>" is indicated instead of a server name.

  - "Tw" is the total time in milliseconds spent waiting in the various queues.
    It can be "-1" if the connection was aborted before reaching the queue.
    See "Timers" below for more details.

  - "Tc" is the total time in milliseconds spent waiting for the connection to
    establish to the final server, including retries. It can be "-1" if the
    connection was aborted before a connection could be established. See
    "Timers" below for more details.

  - "Tt" is the total time in milliseconds elapsed between the accept and the
    last close. It covers all possible processings. There is one exception, if
    "option logasap" was specified, then the time counting stops at the moment
    the log is emitted. In this case, a '+' sign is prepended before the value,
    indicating that the final one will be larger. See "Timers" below for more
    details.

  - "bytes_read" is the total number of bytes transmitted from the server to
    the client when the log is emitted. If "option logasap" is specified, the
    this value will be prefixed with a '+' sign indicating that the final one
    may be larger. Please note that this value is a 64-bit counter, so log
    analysis tools must be able to handle it without overflowing.

  - "termination_state" is the condition the session was in when the session
    ended. This indicates the session state, which side caused the end of
    session to happen, and for what reason (timeout, error, ...). The normal
    flags should be "--", indicating the session was closed by either end with
    no data remaining in buffers. See below "Session state at disconnection"
    for more details.

  - "actconn" is the total number of concurrent connections on the process when
    the session was logged. It it useful to detect when some per-process system
    limits have been reached. For instance, if actconn is close to 512 when
    multiple connection errors occur, chances are high that the system limits
    the process to use a maximum of 1024 file descriptors and that all of them
    are used. See section 3 "Global parameters" to find how to tune the system.

  - "feconn" is the total number of concurrent connections on the frontend when
    the session was logged. It is useful to estimate the amount of resource
    required to sustain high loads, and to detect when the frontend's "maxconn"
    has been reached. Most often when this value increases by huge jumps, it is
    because there is congestion on the backend servers, but sometimes it can be
    caused by a denial of service attack.

  - "beconn" is the total number of concurrent connections handled by the
    backend when the session was logged. It includes the total number of
    concurrent connections active on servers as well as the number of
    connections pending in queues. It is useful to estimate the amount of
    additional servers needed to support high loads for a given application.
    Most often when this value increases by huge jumps, it is because there is
    congestion on the backend servers, but sometimes it can be caused by a
    denial of service attack.

  - "srv_conn" is the total number of concurrent connections still active on
    the server when the session was logged. It can never exceed the server's
    configured "maxconn" parameter. If this value is very often close or equal
    to the server's "maxconn", it means that traffic regulation is involved a
    lot, meaning that either the server's maxconn value is too low, or that
    there aren't enough servers to process the load with an optimal response
    time. When only one of the server's "srv_conn" is high, it usually means
    that this server has some trouble causing the connections to take longer to
    be processed than on other servers.

  - "retries" is the number of connection retries experienced by this session
    when trying to connect to the server. It must normally be zero, unless a
    server is being stopped at the same moment the connection was attempted.
    Frequent retries generally indicate either a network problem between
    haproxy and the server, or a misconfigured system backlog on the server
    preventing new connections from being queued. This field may optionally be
    prefixed with a '+' sign, indicating that the session has experienced a
    redispatch after the maximal retry count has been reached on the initial
    server. In this case, the server name appearing in the log is the one the
    connection was redispatched to, and not the first one, though both may
    sometimes be the same in case of hashing for instance. So as a general rule
    of thumb, when a '+' is present in front of the retry count, this count
    should not be attributed to the logged server.

  - "srv_queue" is the total number of requests which were processed before
    this one in the server queue. It is zero when the request has not gone
    through the server queue. It makes it possible to estimate the approximate
    server's response time by dividing the time spent in queue by the number of
    requests in the queue. It is worth noting that if a session experiences a
    redispatch and passes through two server queues, their positions will be
    cumulated. A request should not pass through both the server queue and the
    backend queue unless a redispatch occurs.

  - "backend_queue" is the total number of requests which were processed before
    this one in the backend's global queue. It is zero when the request has not
    gone through the global queue. It makes it possible to estimate the average
    queue length, which easily translates into a number of missing servers when
    divided by a server's "maxconn" parameter. It is worth noting that if a
    session experiences a redispatch, it may pass twice in the backend's queue,
    and then both positions will be cumulated. A request should not pass
    through both the server queue and the backend queue unless a redispatch
    occurs.


\end{verbatim}
