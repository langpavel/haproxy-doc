% This is generated content
% Section 7.

\chapter{Using ACLs and pattern extraction}

\begin{verbatim}

The use of Access Control Lists (ACL) provides a flexible solution to perform
content switching and generally to take decisions based on content extracted
from the request, the response or any environmental status. The principle is
simple :

  - define test criteria with sets of values
  - perform actions only if a set of tests is valid

The actions generally consist in blocking the request, or selecting a backend.

In order to define a test, the "acl" keyword is used. The syntax is :

   acl <aclname> <criterion> [flags] [operator] <value> ...

This creates a new ACL <aclname> or completes an existing one with new tests.
Those tests apply to the portion of request/response specified in <criterion>
and may be adjusted with optional flags [flags]. Some criteria also support
an operator which may be specified before the set of values. The values are
of the type supported by the criterion, and are separated by spaces.

ACL names must be formed from upper and lower case letters, digits, '-' (dash),
'_' (underscore) , '.' (dot) and ':' (colon). ACL names are case-sensitive,
which means that "my_acl" and "My_Acl" are two different ACLs.

There is no enforced limit to the number of ACLs. The unused ones do not affect
performance, they just consume a small amount of memory.

The following ACL flags are currently supported :

   -i : ignore case during matching of all subsequent patterns.
   -f : load patterns from a file.
   -- : force end of flags. Useful when a string looks like one of the flags.

The "-f" flag is special as it loads all of the lines it finds in the file
specified in argument and loads all of them before continuing. It is even
possible to pass multiple "-f" arguments if the patterns are to be loaded from
multiple files. Empty lines as well as lines beginning with a sharp ('#') will
be ignored. All leading spaces and tabs will be stripped. If it is absolutely
needed to insert a valid pattern beginning with a sharp, just prefix it with a
space so that it is not taken for a comment. Depending on the data type and
match method, haproxy may load the lines into a binary tree, allowing very fast
lookups. This is true for IPv4 and exact string matching. In this case,
duplicates will automatically be removed. Also, note that the "-i" flag applies
to subsequent entries and not to entries loaded from files preceeding it. For
instance :

    acl valid-ua hdr(user-agent) -f exact-ua.lst -i -f generic-ua.lst  test

In this example, each line of "exact-ua.lst" will be exactly matched against
the "user-agent" header of the request. Then each line of "generic-ua" will be
case-insensitively matched. Then the word "test" will be insensitively matched
too.

Note that right now it is difficult for the ACL parsers to report errors, so if
a file is unreadable or unparsable, the most you'll get is a parse error in the
ACL. Thus, file-based ACLs should only be produced by reliable processes.

Supported types of values are :

  - integers or integer ranges
  - strings
  - regular expressions
  - IP addresses and networks


\end{verbatim}
