% This is generated content
% Section 8.8.

\section{Capturing HTTP headers}

\begin{verbatim}

Header captures are useful to track unique request identifiers set by an upper
proxy, virtual host names, user-agents, POST content-length, referrers, etc. In
the response, one can search for information about the response length, how the
server asked the cache to behave, or an object location during a redirection.

Header captures are performed using the "capture request header" and "capture
response header" statements in the frontend. Please consult their definition in
section 4.2 for more details.

It is possible to include both request headers and response headers at the same
time. Non-existent headers are logged as empty strings, and if one header
appears more than once, only its last occurrence will be logged. Request headers
are grouped within braces '{' and '}' in the same order as they were declared,
and delimited with a vertical bar '|' without any space. Response headers
follow the same representation, but are displayed after a space following the
request headers block. These blocks are displayed just before the HTTP request
in the logs.

  Example :
        # This instance chains to the outgoing proxy
        listen proxy-out
            mode http
            option httplog
            option logasap
            log global
            server cache1 192.168.1.1:3128

            # log the name of the virtual server
            capture request  header Host len 20

            # log the amount of data uploaded during a POST
            capture request  header Content-Length len 10

            # log the beginning of the referrer
            capture request  header Referer len 20

            # server name (useful for outgoing proxies only)
            capture response header Server len 20

            # logging the content-length is useful with "option logasap"
            capture response header Content-Length len 10

            # log the expected cache behaviour on the response
            capture response header Cache-Control len 8

            # the Via header will report the next proxy's name
            capture response header Via len 20

            # log the URL location during a redirection
            capture response header Location len 20

    >>> Aug  9 20:26:09 localhost \
          haproxy[2022]: 127.0.0.1:34014 [09/Aug/2004:20:26:09] proxy-out \
          proxy-out/cache1 0/0/0/162/+162 200 +350 - - ---- 0/0/0/0/0 0/0 \
          {fr.adserver.yahoo.co||http://fr.f416.mail.} {|864|private||} \
          "GET http://fr.adserver.yahoo.com/"

    >>> Aug  9 20:30:46 localhost \
          haproxy[2022]: 127.0.0.1:34020 [09/Aug/2004:20:30:46] proxy-out \
          proxy-out/cache1 0/0/0/182/+182 200 +279 - - ---- 0/0/0/0/0 0/0 \
          {w.ods.org||} {Formilux/0.1.8|3495|||} \
          "GET http://trafic.1wt.eu/ HTTP/1.1"

    >>> Aug  9 20:30:46 localhost \
          haproxy[2022]: 127.0.0.1:34028 [09/Aug/2004:20:30:46] proxy-out \
          proxy-out/cache1 0/0/2/126/+128 301 +223 - - ---- 0/0/0/0/0 0/0 \
          {www.sytadin.equipement.gouv.fr||http://trafic.1wt.eu/} \
          {Apache|230|||http://www.sytadin.} \
          "GET http://www.sytadin.equipement.gouv.fr/ HTTP/1.1"


\end{verbatim}
