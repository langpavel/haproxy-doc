% This is generated content
% Section 7.5.1.

\subsection{Matching at Layer 4 and below}

\begin{verbatim}

A first set of criteria applies to information which does not require any
analysis of the request or response contents. Those generally include TCP/IP
addresses and ports, as well as internal values independant on the stream.

always_false
  This one never matches. All values and flags are ignored. It may be used as
  a temporary replacement for another one when adjusting configurations.

always_true
  This one always matches. All values and flags are ignored. It may be used as
  a temporary replacement for another one when adjusting configurations.

avg_queue <integer>
avg_queue(backend) <integer>
  Returns the total number of queued connections of the designated backend
  divided by the number of active servers. This is very similar to "queue"
  except that the size of the farm is considered, in order to give a more
  accurate measurement of the time it may take for a new connection to be
  processed. The main usage is to return a sorry page to new users when it
  becomes certain they will get a degraded service. Note that in the event
  there would not be any active server anymore, we would consider twice the
  number of queued connections as the measured value. This is a fair estimate,
  as we expect one server to get back soon anyway, but we still prefer to send
  new traffic to another backend if in better shape. See also the "queue",
  "be_conn", and "be_sess_rate" criteria.

be_conn <integer>
be_conn(backend) <integer>
  Applies to the number of currently established connections on the backend,
  possibly including the connection being evaluated. If no backend name is
  specified, the current one is used. But it is also possible to check another
  backend. It can be used to use a specific farm when the nominal one is full.
  See also the "fe_conn", "queue" and "be_sess_rate" criteria.

be_id <integer>
  Applies to the backend's id. Can be used in frontends to check from which
  backend it was called.

be_sess_rate <integer>
be_sess_rate(backend) <integer>
  Returns true when the sessions creation rate on the backend matches the
  specified values or ranges, in number of new sessions per second. This is
  used to switch to an alternate backend when an expensive or fragile one
  reaches too high a session rate, or to limit abuse of service (eg. prevent
  sucking of an online dictionary).

  Example :
        # Redirect to an error page if the dictionary is requested too often
        backend dynamic
            mode http
            acl being_scanned be_sess_rate gt 100
            redirect location /denied.html if being_scanned

connslots <integer>
connslots(backend) <integer>
  The basic idea here is to be able to measure the number of connection "slots"
  still available (connection + queue), so that anything beyond that (intended
  usage; see "use_backend" keyword) can be redirected to a different backend.

  'connslots' = number of available server connection slots, + number of
  available server queue slots.

  Note that while "fe_conn" may be used, "connslots" comes in especially
  useful when you have a case of traffic going to one single ip, splitting into
  multiple backends (perhaps using acls to do name-based load balancing) and
  you want to be able to differentiate between different backends, and their
  available "connslots".  Also, whereas "nbsrv" only measures servers that are
  actually *down*, this acl is more fine-grained and looks into the number of
  available connection slots as well. See also "queue" and "avg_queue".

  OTHER CAVEATS AND NOTES: at this point in time, the code does not take care
  of dynamic connections. Also, if any of the server maxconn, or maxqueue is 0,
  then this acl clearly does not make sense, in which case the value returned
  will be -1.

dst <ip_address>
  Applies to the local IPv4 address the client connected to. It can be used to
  switch to a different backend for some alternative addresses.

dst_conn <integer>
  Applies to the number of currently established connections on the same socket
  including the one being evaluated. It can be used to either return a sorry
  page before hard-blocking, or to use a specific backend to drain new requests
  when the socket is considered saturated. This offers the ability to assign
  different limits to different listening ports or addresses. See also the
  "fe_conn" and "be_conn" criteria.

dst_port <integer>
  Applies to the local port the client connected to. It can be used to switch
  to a different backend for some alternative ports.

fe_conn <integer>
fe_conn(frontend) <integer>
  Applies to the number of currently established connections on the frontend,
  possibly including the connection being evaluated. If no frontend name is
  specified, the current one is used. But it is also possible to check another
  frontend. It can be used to either return a sorry page before hard-blocking,
  or to use a specific backend to drain new requests when the farm is
  considered saturated. See also the "dst_conn", "be_conn" and "fe_sess_rate"
  criteria.

fe_id <integer>
  Applies to the frontend's id. Can be used in backends to check from which
  frontend it was called.

fe_sess_rate <integer>
fe_sess_rate(frontend) <integer>
  Returns true when the session creation rate on the current or the named
  frontend matches the specified values or ranges, expressed in new sessions
  per second. This is used to limit the connection rate to acceptable ranges in
  order to prevent abuse of service at the earliest moment. This can be
  combined with layer 4 ACLs in order to force the clients to wait a bit for
  the rate to go down below the limit.

  Example :
        # This frontend limits incoming mails to 10/s with a max of 100
        # concurrent connections. We accept any connection below 10/s, and
        # force excess clients to wait for 100 ms. Since clients are limited to
        # 100 max, there cannot be more than 10 incoming mails per second.
        frontend mail
            bind :25
            mode tcp
            maxconn 100
            acl too_fast fe_sess_rate ge 10
            tcp-request inspect-delay 100ms
            tcp-request content accept if ! too_fast
            tcp-request content accept if WAIT_END

nbsrv <integer>
nbsrv(backend) <integer>
  Returns true when the number of usable servers of either the current backend
  or the named backend matches the values or ranges specified. This is used to
  switch to an alternate backend when the number of servers is too low to
  to handle some load. It is useful to report a failure when combined with
  "monitor fail".

queue <integer>
queue(backend) <integer>
  Returns the total number of queued connections of the designated backend,
  including all the connections in server queues. If no backend name is
  specified, the current one is used, but it is also possible to check another
  one. This can be used to take actions when queuing goes above a known level,
  generally indicating a surge of traffic or a massive slowdown on the servers.
  One possible action could be to reject new users but still accept old ones.
  See also the "avg_queue", "be_conn", and "be_sess_rate" criteria.

so_id <integer>
  Applies to the socket's id. Useful in frontends with many bind keywords.

src <ip_address>
  Applies to the client's IPv4 address. It is usually used to limit access to
  certain resources such as statistics. Note that it is the TCP-level source
  address which is used, and not the address of a client behind a proxy.

src_port <integer>
  Applies to the client's TCP source port. This has a very limited usage.

srv_id <integer>
  Applies to the server's id. Can be used in frontends or backends.

srv_is_up(<server>)
srv_is_up(<backend>/<server>)
  Returns true when the designated server is UP, and false when it is either
  DOWN or in maintenance mode. If <backend> is omitted, then the server is
  looked up in the current backend. The function takes no arguments since it
  is used as a boolean. It is mainly used to take action based on an external
  status reported via a health check (eg: a geographical site's availability).
  Another possible use which is more of a hack consists in using dummy servers
  as boolean variables that can be enabled or disabled from the CLI, so that
  rules depending on those ACLs can be tweaked in realtime.


\end{verbatim}
