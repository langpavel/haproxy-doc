% This is generated content
% Section 8.4.

\section{Timing events}

\begin{verbatim}

Timers provide a great help in troubleshooting network problems. All values are
reported in milliseconds (ms). These timers should be used in conjunction with
the session termination flags. In TCP mode with "option tcplog" set on the
frontend, 3 control points are reported under the form "Tw/Tc/Tt", and in HTTP
mode, 5 control points are reported under the form "Tq/Tw/Tc/Tr/Tt" :

  - Tq: total time to get the client request (HTTP mode only). It's the time
    elapsed between the moment the client connection was accepted and the
    moment the proxy received the last HTTP header. The value "-1" indicates
    that the end of headers (empty line) has never been seen. This happens when
    the client closes prematurely or times out.

  - Tw: total time spent in the queues waiting for a connection slot. It
    accounts for backend queue as well as the server queues, and depends on the
    queue size, and the time needed for the server to complete previous
    requests. The value "-1" means that the request was killed before reaching
    the queue, which is generally what happens with invalid or denied requests.

  - Tc: total time to establish the TCP connection to the server. It's the time
    elapsed between the moment the proxy sent the connection request, and the
    moment it was acknowledged by the server, or between the TCP SYN packet and
    the matching SYN/ACK packet in return. The value "-1" means that the
    connection never established.

  - Tr: server response time (HTTP mode only). It's the time elapsed between
    the moment the TCP connection was established to the server and the moment
    the server sent its complete response headers. It purely shows its request
    processing time, without the network overhead due to the data transmission.
    It is worth noting that when the client has data to send to the server, for
    instance during a POST request, the time already runs, and this can distort
    apparent response time. For this reason, it's generally wise not to trust
    too much this field for POST requests initiated from clients behind an
    untrusted network. A value of "-1" here means that the last the response
    header (empty line) was never seen, most likely because the server timeout
    stroke before the server managed to process the request.

  - Tt: total session duration time, between the moment the proxy accepted it
    and the moment both ends were closed. The exception is when the "logasap"
    option is specified. In this case, it only equals (Tq+Tw+Tc+Tr), and is
    prefixed with a '+' sign. From this field, we can deduce "Td", the data
    transmission time, by substracting other timers when valid :

        Td = Tt - (Tq + Tw + Tc + Tr)

    Timers with "-1" values have to be excluded from this equation. In TCP
    mode, "Tq" and "Tr" have to be excluded too. Note that "Tt" can never be
    negative.

These timers provide precious indications on trouble causes. Since the TCP
protocol defines retransmit delays of 3, 6, 12... seconds, we know for sure
that timers close to multiples of 3s are nearly always related to lost packets
due to network problems (wires, negotiation, congestion). Moreover, if "Tt" is
close to a timeout value specified in the configuration, it often means that a
session has been aborted on timeout.

Most common cases :

  - If "Tq" is close to 3000, a packet has probably been lost between the
    client and the proxy. This is very rare on local networks but might happen
    when clients are on far remote networks and send large requests. It may
    happen that values larger than usual appear here without any network cause.
    Sometimes, during an attack or just after a resource starvation has ended,
    haproxy may accept thousands of connections in a few milliseconds. The time
    spent accepting these connections will inevitably slightly delay processing
    of other connections, and it can happen that request times in the order of
    a few tens of milliseconds are measured after a few thousands of new
    connections have been accepted at once. Setting "option http-server-close"
    may display larger request times since "Tq" also measures the time spent
    waiting for additional requests.

  - If "Tc" is close to 3000, a packet has probably been lost between the
    server and the proxy during the server connection phase. This value should
    always be very low, such as 1 ms on local networks and less than a few tens
    of ms on remote networks.

  - If "Tr" is nearly always lower than 3000 except some rare values which seem
    to be the average majored by 3000, there are probably some packets lost
    between the proxy and the server.

  - If "Tt" is large even for small byte counts, it generally is because
    neither the client nor the server decides to close the connection, for
    instance because both have agreed on a keep-alive connection mode. In order
    to solve this issue, it will be needed to specify "option httpclose" on
    either the frontend or the backend. If the problem persists, it means that
    the server ignores the "close" connection mode and expects the client to
    close. Then it will be required to use "option forceclose". Having the
    smallest possible 'Tt' is important when connection regulation is used with
    the "maxconn" option on the servers, since no new connection will be sent
    to the server until another one is released.

Other noticeable HTTP log cases ('xx' means any value to be ignored) :

  Tq/Tw/Tc/Tr/+Tt  The "option logasap" is present on the frontend and the log
                   was emitted before the data phase. All the timers are valid
                   except "Tt" which is shorter than reality.

  -1/xx/xx/xx/Tt   The client was not able to send a complete request in time
                   or it aborted too early. Check the session termination flags
                   then "timeout http-request" and "timeout client" settings.

  Tq/-1/xx/xx/Tt   It was not possible to process the request, maybe because
                   servers were out of order, because the request was invalid
                   or forbidden by ACL rules. Check the session termination
                   flags.

  Tq/Tw/-1/xx/Tt   The connection could not establish on the server. Either it
                   actively refused it or it timed out after Tt-(Tq+Tw) ms.
                   Check the session termination flags, then check the
                   "timeout connect" setting. Note that the tarpit action might
                   return similar-looking patterns, with "Tw" equal to the time
                   the client connection was maintained open.

  Tq/Tw/Tc/-1/Tt   The server has accepted the connection but did not return
                   a complete response in time, or it closed its connexion
                   unexpectedly after Tt-(Tq+Tw+Tc) ms. Check the session
                   termination flags, then check the "timeout server" setting.


\end{verbatim}
