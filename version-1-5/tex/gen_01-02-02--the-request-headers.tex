% This is generated content
% Section 1.2.2.

\subsection{The request headers}

The headers start at the second line. They are composed of a name at the
beginning of the line, immediately followed by a colon \CHAR{:}. Traditionally,
an LWS is added after the colon but that's not required. Then come the values.
Multiple identical headers may be folded into one single line, delimiting the
values with commas, provided that their order is respected. This is commonly
encountered in the "Cookie:" field. A header may span over multiple lines if
the subsequent lines begin with an LWS. In the example in 1.2, lines 4 and 5
define a total of 3 values for the "Accept:" header.


Contrary to a common mis-conception, header names are not case-sensitive, and
their values are not either if they refer to other header names (such as the
"Connection:" header).


The end of the headers is indicated by the first empty line. People often say
that it's a double line feed, which is not exact, even if a double line feed
is one valid form of empty line.


Fortunately, HAProxy takes care of all these complex combinations when indexing
headers, checking values and counting them, so there is no reason to worry
about the way they could be written, but it is important not to accuse an
application of being buggy if it does unusual, valid things.

\begin{verbatim}
Important note:
   As suggested by RFC2616, HAProxy normalizes headers by replacing line breaks
   in the middle of headers by LWS in order to join multi-line headers. This
   is necessary for proper analysis and helps less capable HTTP parsers to work
   correctly and not to be fooled by such complex constructs.
\end{verbatim}

