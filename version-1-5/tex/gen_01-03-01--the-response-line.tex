% This is generated content
% Section 1.3.1.

\subsection{The Response line}

Line 1 is the \emph{response line}. It is always composed of 3 fields :

\begin{verbatim}
  - a version tag : HTTP/1.1
  - a status code : 200
  - a reason      : OK
\end{verbatim}


The status code is always 3-digit. The first digit indicates a general status :
 - 1xx = informational message to be skipped (eg: 100, 101)
 - 2xx = OK, content is following   (eg: 200, 206)
 - 3xx = OK, no content following   (eg: 302, 304)
 - 4xx = error caused by the client (eg: 401, 403, 404)
 - 5xx = error caused by the server (eg: 500, 502, 503)


Please refer to RFC2616 for the detailed meaning of all such codes. The
\emph{reason} field is just a hint, but is not parsed by clients. Anything can be
found there, but it's a common practice to respect the well-established
messages. It can be composed of one or multiple words, such as \emph{OK}, \emph{Found},
or \emph{Authentication Required}.


Haproxy may emit the following status codes by itself :

\begin{verbatim}
  Code  When / reason
   200  access to stats page, and when replying to monitoring requests
   301  when performing a redirection, depending on the configured code
   302  when performing a redirection, depending on the configured code
   303  when performing a redirection, depending on the configured code
   400  for an invalid or too large request
   401  when an authentication is required to perform the action (when
        accessing the stats page)
   403  when a request is forbidden by a "block" ACL or "reqdeny" filter
   408  when the request timeout strikes before the request is complete
   500  when haproxy encounters an unrecoverable internal error, such as a
        memory allocation failure, which should never happen
   502  when the server returns an empty, invalid or incomplete response, or
        when an "rspdeny" filter blocks the response.
   503  when no server was available to handle the request, or in response to
        monitoring requests which match the "monitor fail" condition
   504  when the response timeout strikes before the server responds
\end{verbatim}


The error 4xx and 5xx codes above may be customized (see \emph{errorloc} in section
4.2).

