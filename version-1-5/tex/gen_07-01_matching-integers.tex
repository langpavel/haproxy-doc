% This is generated content
% Section 7.1.

\section{Matching integers}

\begin{verbatim}

Matching integers is special in that ranges and operators are permitted. Note
that integer matching only applies to positive values. A range is a value
expressed with a lower and an upper bound separated with a colon, both of which
may be omitted.

For instance, "1024:65535" is a valid range to represent a range of
unprivileged ports, and "1024:" would also work. "0:1023" is a valid
representation of privileged ports, and ":1023" would also work.

As a special case, some ACL functions support decimal numbers which are in fact
two integers separated by a dot. This is used with some version checks for
instance. All integer properties apply to those decimal numbers, including
ranges and operators.

For an easier usage, comparison operators are also supported. Note that using
operators with ranges does not make much sense and is strongly discouraged.
Similarly, it does not make much sense to perform order comparisons with a set
of values.

Available operators for integer matching are :

  eq : true if the tested value equals at least one value
  ge : true if the tested value is greater than or equal to at least one value
  gt : true if the tested value is greater than at least one value
  le : true if the tested value is less than or equal to at least one value
  lt : true if the tested value is less than at least one value

For instance, the following ACL matches any negative Content-Length header :

  acl negative-length hdr_val(content-length) lt 0

This one matches SSL versions between 3.0 and 3.1 (inclusive) :

  acl sslv3 req_ssl_ver 3:3.1


\end{verbatim}
