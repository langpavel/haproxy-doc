% Section 1.3.
\pagebreak[4]
\section{HTTP response}

An HTTP response looks very much like an HTTP request. Both are called HTTP
messages. Let's consider this HTTP response:

\label{HTTP response}\begin{listing}{1}
HTTP/1.1 200 OK
Content-length: 350
Content-Type: text/html
\end{listing}

As a special case, HTTP supports so called \emph{Informational responses} as status
codes 1xx. These messages are special in that they don't convey any part of the
response, they're just used as sort of a signaling message to ask a client to
continue to post its request for instance. In the case of a status 100 response
the requested information will be carried by the next non-100 response message
following the informational one. This implies that multiple responses may be
sent to a single request, and that this only works when keep-alive is enabled
(1xx messages are HTTP/1.1 only). HAProxy handles these messages and is able to
correctly forward and skip them, and only process the next non-100 response. As
such, these messages are neither logged nor transformed, unless explicitly
state otherwise. Status 101 messages indicate that the protocol is changing
over the same connection and that haproxy must switch to tunnel mode, just as
if a \httpmethod{CONNECT} had occurred. Then the Upgrade header would contain additional
information about the type of protocol the connection is switching to.


