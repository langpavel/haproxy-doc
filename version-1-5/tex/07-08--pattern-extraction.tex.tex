% This is generated content
% Section 7.8.

\section{Pattern extraction}

\begin{verbatim}

The stickiness features relies on pattern extraction in the request and
response. Sometimes the data needs to be converted first before being stored,
for instance converted from ASCII to IP or upper case to lower case.

All these operations of data extraction and conversion are defined as
"pattern extraction rules". A pattern rule always has the same format. It
begins with a single pattern fetch word, potentially followed by a list of
arguments within parenthesis then an optional list of transformations. As
much as possible, the pattern fetch functions use the same name as their
equivalent used in ACLs.

The list of currently supported pattern fetch functions is the following :

  src          This is the source IPv4 address of the client of the session.
               It is of type IPv4 and works on both IPv4 and IPv6 tables.
               On IPv6 tables, IPv4 address is mapped to its IPv6 equivalent,
               according to RFC 4291.

  src6         This is the source IPv6 address of the client of the session.
               It is of type IPv6 and only works with such tables.

  dst          This is the destination IPv4 address of the session on the
               client side, which is the address the client connected to.
               It can be useful when running in transparent mode. It is of
               type IPv4 and works on both IPv4 and IPv6 tables.
               On IPv6 tables, IPv4 address is mapped to its IPv6 equivalent,
               according to RFC 4291.

  dst6         This is the destination IPv6 address of the session on the
               client side, which is the address the client connected to.
               It can be useful when running in transparent mode. It is of
               type IPv6 and only works with such tables.

  dst_port     This is the destination TCP port of the session on the client
               side, which is the port the client connected to. This might be
               used when running in transparent mode or when assigning dynamic
               ports to some clients for a whole application session. It is of
               type integer and only works with such tables.

  hdr(<name>)    This extracts the last occurrence of header <name> in an HTTP
               request and converts it to an IP address. This IP address is
               then used to match the table. A typical use is with the
               x-forwarded-for header.

  payload(<offset>,<length>)
               This extracts a binary block of <length> bytes, and starting
               at bytes <offset> in the buffer of request or response (request
               on "stick on" or "stick match" or response in on "stick store
               response").

  payload_lv(<offset1>,<length>[,<offset2>])
               This extracts a binary block. In a first step the size of the
               block is read from response or request buffer at <offset>
               bytes and considered coded on <length> bytes. In a second step
               data of the block are read from buffer at <offset2> bytes
               (by default <lengthoffset> + <lengthsize>).
               If <offset2> is prefixed by '+' or '-', it is relative to
               <lengthoffset> + <lengthsize> else it is absolute.
               Ex: see SSL session id  example in "stick table" chapter.
                
  url_param(<name>)
	       This extracts the first occurrence of the parameter <name> in
	       the query string of the request and uses the corresponding value
               to match. A typical use is to get sticky session through url (e.g.
	       http://example.com/foo?JESSIONID=some_id with
	       url_param(JSESSIONID)), for cases where cookies cannot be used.

  rdp_cookie(<name>)
	       This extracts the value of the rdp cookie <name> as a string
	       and uses this value to match. This enables implementation of
	       persistence based on the mstshash cookie. This is typically
	       done if there is no msts cookie present.

	       This differs from "balance rdp-cookie" in that any balancing
	       algorithm may be used and thus the distribution of clients
	       to backend servers is not linked to a hash of the RDP
	       cookie. It is envisaged that using a balancing algorithm
	       such as "balance roundrobin" or "balance leastconnect" will
	       lead to a more even distribution of clients to backend
	       servers than the hash used by "balance rdp-cookie".

	       Example :
		listen tse-farm
		    bind 0.0.0.0:3389
		    # wait up to 5s for an RDP cookie in the request
		    tcp-request inspect-delay 5s
		    tcp-request content accept if RDP_COOKIE
		    # apply RDP cookie persistence
		    persist rdp-cookie
		    # Persist based on the mstshash cookie
		    # This is only useful makes sense if
		    # balance rdp-cookie is not used
		    stick-table type string size 204800
		    stick on rdp_cookie(mstshash)
		    server srv1 1.1.1.1:3389
		    server srv1 1.1.1.2:3389

	       See also : "balance rdp-cookie", "persist rdp-cookie",
	       "tcp-request" and the "req_rdp_cookie" ACL.

  cookie(<name>)
	       This extracts the last occurrence of the cookie name <name> on a
	       "Cookie" header line from the request and uses the corresponding
	       value to match. A typical use is to get multiple clients sharing
	       a same profile use the same server. This can be similar to what
	       "appsession" does with the "request-learn" statement, but with
	       support for multi-peer synchronization and state keeping across
	       restarts.

	       See also : "appsession"

  set-cookie(<name>)
	       This extracts the last occurrence of the cookie name <name> on a
	       "Set-Cookie" header line from the response and uses the
	       corresponding value to match. This can be comparable to what
	       "appsession" does with default options, but with support for
	        multi-peer synchronization and state keeping across restarts.

	       See also : "appsession"


The currently available list of transformations include :

  lower        Convert a string pattern to lower case. This can only be placed
               after a string pattern fetch function or after a conversion
               function returning a string type. The result is of type string.

  upper        Convert a string pattern to upper case. This can only be placed
               after a string pattern fetch function or after a conversion
               function returning a string type. The result is of type string.

  ipmask(<mask>) Apply a mask to an IPv4 address, and use the result for lookups
               and storage. This can be used to make all hosts within a
               certain mask to share the same table entries and as such use
               the same server. The mask can be passed in dotted form (eg:
               255.255.255.0) or in CIDR form (eg: 24).


\end{verbatim}
