% This is generated content
% Section 1.1.

\section{The HTTP transaction model}

The HTTP protocol is transaction-driven. This means that each request will lead
to one and only one response. Traditionally, a TCP connection is established
from the client to the server, a request is sent by the client on the
connection, the server responds and the connection is closed. A new request
will involve a new connection :

\begin{verbatim}
  [CON1] [REQ1] ... [RESP1] [CLO1] [CON2] [REQ2] ... [RESP2] [CLO2] ...
\end{verbatim}


In this mode, called the \emph{HTTP close} mode, there are as many connection
establishments as there are HTTP transactions. Since the connection is closed
by the server after the response, the client does not need to know the content
length.


Due to the transactional nature of the protocol, it was possible to improve it
to avoid closing a connection between two subsequent transactions. In this mode
however, it is mandatory that the server indicates the content length for each
response so that the client does not wait indefinitely. For this, a special
header is used: "Content-length". This mode is called the "keep-alive" mode :

\begin{verbatim}
  [CON] [REQ1] ... [RESP1] [REQ2] ... [RESP2] [CLO] ...
\end{verbatim}


Its advantages are a reduced latency between transactions, and less processing
power required on the server side. It is generally better than the close mode,
but not always because the clients often limit their concurrent connections to
a smaller value.


A last improvement in the communications is the pipelining mode. It still uses
keep-alive, but the client does not wait for the first response to send the
second request. This is useful for fetching large number of images composing a
page :

\begin{verbatim}
  [CON] [REQ1] [REQ2] ... [RESP1] [RESP2] [CLO] ...
\end{verbatim}


This can obviously have a tremendous benefit on performance because the network
latency is eliminated between subsequent requests. Many HTTP agents do not
correctly support pipelining since there is no way to associate a response with
the corresponding request in HTTP. For this reason, it is mandatory for the
server to reply in the exact same order as the requests were received.


By default HAProxy operates in a tunnel-like mode with regards to persistent
connections: for each connection it processes the first request and forwards
everything else (including additional requests) to selected server. Once
established, the connection is persisted both on the client and server
sides. Use "option http-server-close" to preserve client persistent connections
while handling every incoming request individually, dispatching them one after
another to servers, in HTTP close mode. Use \emph{option httpclose} to switch both
sides to HTTP close mode. \emph{option forceclose} and "option
http-pretend-keepalive" help working around servers misbehaving in HTTP close
mode.

