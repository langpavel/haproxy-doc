% Section 3.4.

\section{Userlists}

It is possible to control access to 
\keyword{frontend}/\keyword{backend}/\keyword{listen} sections or to
http stats by allowing only authenticated and authorized users. To do this,
it is required to create at least one userlist and to define users.

\begin{keywords}

\kw{userlist} \param{listname}

  Creates new userlist with name \param{listname}. Many independent userlists can be
  used to store authentication \& authorization data for independent customers.

\kw{group} \param{groupname}
	\oparam{users~\param{user},\param{user},(\dots )}

  Adds group <groupname> to the current userlist. It is also possible to
  attach users to this group by using a comma separated list of names
  proceeded by \keyword{users} keyword.

\kw{user} \param{username}
	\oparam{password|insecure-password~\param{password}}
	\oparam{groups~\param{group},\param{group},(\dots )}

  Adds user \param{username} to the current userlist. Both secure (encrypted) and
  insecure (unencrypted) passwords can be used. Encrypted passwords are
  evaluated using the crypt(3) function so depending of the system's
  capabilities, different algorithms are supported. For example modern Glibc
  based Linux system supports MD5, SHA-256, SHA-512 and of course classic,
  DES-based method of crypting passwords.


  Example:
\begin{listing}{1}
userlist L1
  group G1 users tiger,scott
  group G2 users xdb,scott

  user tiger password $6$k6y3o.eP$JlKBx9za9667qe4(...)xHSwRv6J.C0/D7cV91
  user scott insecure-password elgato
  user xdb insecure-password hello

userlist L2
  group G1
  group G2

  user tiger password $6$k6y3o.eP$JlKBx(...)xHSwRv6J.C0/D7cV91 groups G1
  user scott insecure-password elgato groups G1,G2
  user xdb insecure-password hello groups G2
\end{listing}

\note Note:
that both lists are functionally identical.

\end{keywords}

