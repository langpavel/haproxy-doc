% This is generated content
% Section 3.1.

\section{Process management and security}
\begin{verbatim}
chroot <jail dir>
  Changes current directory to <jail dir> and performs a chroot() there before
  dropping privileges. This increases the security level in case an unknown
  vulnerability would be exploited, since it would make it very hard for the
  attacker to exploit the system. This only works when the process is started
  with superuser privileges. It is important to ensure that <jail_dir> is both
  empty and unwritable to anyone.
\end{verbatim}

\begin{verbatim}
  Makes the process fork into background. This is the recommended mode of
  operation. It is equivalent to the command line "-D" argument. It can be
  disabled by the command line "-db" argument.
\end{verbatim}

\begin{verbatim}
gid <number>
  Changes the process' group ID to <number>. It is recommended that the group
  ID is dedicated to HAProxy or to a small set of similar daemons. HAProxy must
  be started with a user belonging to this group, or with superuser privileges.
  See also "group" and "uid".
\end{verbatim}

\begin{verbatim}
group <group name>
  Similar to "gid" but uses the GID of group name <group name> from /etc/group.
  See also "gid" and "user".
\end{verbatim}

\begin{verbatim}
log <address> <facility> [max level [min level]]
  Adds a global syslog server. Up to two global servers can be defined. They
  will receive logs for startups and exits, as well as all logs from proxies
  configured with "log global".
\end{verbatim}

\begin{verbatim}
  <address> can be one of:
\end{verbatim}

\begin{verbatim}
        - An IPv4 address optionally followed by a colon and a UDP port. If
          no port is specified, 514 is used by default (the standard syslog
          port).
\end{verbatim}

\begin{verbatim}
        - An IPv6 address followed by a colon and optionally a UDP port. If
          no port is specified, 514 is used by default (the standard syslog
          port).
\end{verbatim}

\begin{verbatim}
        - A filesystem path to a UNIX domain socket, keeping in mind
          considerations for chroot (be sure the path is accessible inside
          the chroot) and uid/gid (be sure the path is appropriately
          writeable).
\end{verbatim}

\begin{verbatim}
  <facility> must be one of the 24 standard syslog facilities :
\end{verbatim}

\begin{verbatim}
          kern   user   mail   daemon auth   syslog lpr    news
          uucp   cron   auth2  ftp    ntp    audit  alert  cron2
          local0 local1 local2 local3 local4 local5 local6 local7
\end{verbatim}

\begin{verbatim}
  An optional level can be specified to filter outgoing messages. By default,
  all messages are sent. If a maximum level is specified, only messages with a
  severity at least as important as this level will be sent. An optional minimum
  level can be specified. If it is set, logs emitted with a more severe level
  than this one will be capped to this level. This is used to avoid sending
  "emerg" messages on all terminals on some default syslog configurations.
  Eight levels are known :
\end{verbatim}

\begin{verbatim}
          emerg  alert  crit   err    warning notice info  debug
\end{verbatim}

\begin{verbatim}
log-send-hostname [<string>]
  Sets the hostname field in the syslog header. If optional "string" parameter
  is set the header is set to the string contents, otherwise uses the hostname
  of the system. Generally used if one is not relaying logs through an
  intermediate syslog server or for simply customizing the hostname printed in
  the logs.
\end{verbatim}

\begin{verbatim}
log-tag <string>
  Sets the tag field in the syslog header to this string. It defaults to the
  program name as launched from the command line, which usually is "haproxy".
  Sometimes it can be useful to differentiate between multiple processes
  running on the same host.
\end{verbatim}

\begin{verbatim}
nbproc <number>
  Creates <number> processes when going daemon. This requires the "daemon"
  mode. By default, only one process is created, which is the recommended mode
  of operation. For systems limited to small sets of file descriptors per
  process, it may be needed to fork multiple daemons. USING MULTIPLE PROCESSES
  IS HARDER TO DEBUG AND IS REALLY DISCOURAGED. See also "daemon".
\end{verbatim}

\begin{verbatim}
pidfile <pidfile>
  Writes pids of all daemons into file <pidfile>. This option is equivalent to
  the "-p" command line argument. The file must be accessible to the user
  starting the process. See also "daemon".
\end{verbatim}

\begin{verbatim}
stats socket <path> [{uid | user} <uid>] [{gid | group} <gid>] [mode <mode>]
  [level <level>]
\end{verbatim}

\begin{verbatim}
  Creates a UNIX socket in stream mode at location <path>. Any previously
  existing socket will be backed up then replaced. Connections to this socket
  will return various statistics outputs and even allow some commands to be
  issued. Please consult section 9.2 "Unix Socket commands" for more details.
\end{verbatim}

\begin{verbatim}
  An optional "level" parameter can be specified to restrict the nature of
  the commands that can be issued on the socket :
    - "user" is the least privileged level ; only non-sensitive stats can be
      read, and no change is allowed. It would make sense on systems where it
      is not easy to restrict access to the socket.
\end{verbatim}

\begin{verbatim}
    - "operator" is the default level and fits most common uses. All data can
      be read, and only non-sensitive changes are permitted (eg: clear max
      counters).
\end{verbatim}

\begin{verbatim}
    - "admin" should be used with care, as everything is permitted (eg: clear
      all counters).
\end{verbatim}

\begin{verbatim}
  On platforms which support it, it is possible to restrict access to this
  socket by specifying numerical IDs after "uid" and "gid", or valid user and
  group names after the "user" and "group" keywords. It is also possible to
  restrict permissions on the socket by passing an octal value after the "mode"
  keyword (same syntax as chmod). Depending on the platform, the permissions on
  the socket will be inherited from the directory which hosts it, or from the
  user the process is started with.
\end{verbatim}

\begin{verbatim}
stats timeout <timeout, in milliseconds>
  The default timeout on the stats socket is set to 10 seconds. It is possible
  to change this value with "stats timeout". The value must be passed in
  milliseconds, or be suffixed by a time unit among { us, ms, s, m, h, d }.
\end{verbatim}

\begin{verbatim}
stats maxconn <connections>
  By default, the stats socket is limited to 10 concurrent connections. It is
  possible to change this value with "stats maxconn".
\end{verbatim}

\begin{verbatim}
uid <number>
  Changes the process' user ID to <number>. It is recommended that the user ID
  is dedicated to HAProxy or to a small set of similar daemons. HAProxy must
  be started with superuser privileges in order to be able to switch to another
  one. See also "gid" and "user".
\end{verbatim}

\begin{verbatim}
ulimit-n <number>
  Sets the maximum number of per-process file-descriptors to <number>. By
  default, it is automatically computed, so it is recommended not to use this
  option.
\end{verbatim}

\begin{verbatim}
unix-bind [ prefix <prefix> ] [ mode <mode> ] [ user <user> ] [ uid <uid> ]
          [ group <group> ] [ gid <gid> ]
\end{verbatim}

\begin{verbatim}
  Fixes common settings to UNIX listening sockets declared in "bind" statements.
  This is mainly used to simplify declaration of those UNIX sockets and reduce
  the risk of errors, since those settings are most commonly required but are
  also process-specific. The <prefix> setting can be used to force all socket
  path to be relative to that directory. This might be needed to access another
  component's chroot. Note that those paths are resolved before haproxy chroots
  itself, so they are absolute. The <mode>, <user>, <uid>, <group> and <gid>
  all have the same meaning as their homonyms used by the "bind" statement. If
  both are specified, the "bind" statement has priority, meaning that the
  "unix-bind" settings may be seen as process-wide default settings.
\end{verbatim}

\begin{verbatim}
user <user name>
  Similar to "uid" but uses the UID of user name <user name> from /etc/passwd.
  See also "uid" and "group".
\end{verbatim}

\begin{verbatim}
node <name>
  Only letters, digits, hyphen and underscore are allowed, like in DNS names.
\end{verbatim}

\begin{verbatim}
  This statement is useful in HA configurations where two or more processes or
  servers share the same IP address. By setting a different node-name on all
  nodes, it becomes easy to immediately spot what server is handling the
  traffic.
\end{verbatim}

\begin{verbatim}
description <text>
  Add a text that describes the instance.
\end{verbatim}

\begin{verbatim}
  Please note that it is required to escape certain characters (# for example)
  and this text is inserted into a html page so you should avoid using
  "<" and ">" characters.
\end{verbatim}

