% This is generated content
% Section Summary

\chapter*{Summary}
\begin{verbatim}
                         ----------------------
                                 HAProxy
                          Configuration Manual
                         ----------------------
                             version 1.5
                             willy tarreau
                              2011/09/10
\end{verbatim}


This document covers the configuration language as implemented in the version
specified above. It does not provide any hint, example or advice. For such
documentation, please refer to the Reference Manual or the Architecture Manual.
The summary below is meant to help you search sections by name and navigate
through the document.

\begin{verbatim}
Note to documentation contributors :
    This document is formated with 80 columns per line, with even number of
    spaces for indentation and without tabs. Please follow these rules strictly
    so that it remains easily printable everywhere. If a line needs to be
    printed verbatim and does not fit, please end each line with a backslash
    ('\') and continue on next line, indented by two characters. It is also
    sometimes useful to prefix all output lines (logs, console outs) with 3
    closing angle brackets ('>>>') in order to help get the difference between
    inputs and outputs when it can become ambiguous. If you add sections,
    please update the summary below for easier searching.
\end{verbatim}

\begin{verbatim}
-------
\end{verbatim}


1.    Quick reminder about HTTP
1.1.      The HTTP transaction model
1.2.      HTTP request
1.2.1.        The Request line
1.2.2.        The request headers
1.3.      HTTP response
1.3.1.        The Response line
1.3.2.        The response headers


2.    Configuring HAProxy
2.1.      Configuration file format
2.2.      Time format
2.3.      Examples


3.    Global parameters
3.1.      Process management and security
3.2.      Performance tuning
3.3.      Debugging
3.4.      Userlists


4.    Proxies
4.1.      Proxy keywords matrix
4.2.      Alphabetically sorted keywords reference


5.    Server and default-server options


6.    HTTP header manipulation


7.    Using ACLs and pattern extraction
7.1.      Matching integers
7.2.      Matching strings
7.3.      Matching regular expressions (regexes)
7.4.      Matching IPv4 addresses
7.5.      Available matching criteria
7.5.1.        Matching at Layer 4 and below
7.5.2.        Matching contents at Layer 4
7.5.3.        Matching at Layer 7
7.6.      Pre-defined ACLs
7.7.      Using ACLs to form conditions
7.8.      Pattern extraction


8.    Logging
8.1.      Log levels
8.2.      Log formats
8.2.1.        Default log format
8.2.2.        TCP log format
8.2.3.        HTTP log format
8.3.      Advanced logging options
8.3.1.        Disabling logging of external tests
8.3.2.        Logging before waiting for the session to terminate
8.3.3.        Raising log level upon errors
8.3.4.        Disabling logging of successful connections
8.4.      Timing events
8.5.      Session state at disconnection
8.6.      Non-printable characters
8.7.      Capturing HTTP cookies
8.8.      Capturing HTTP headers
8.9.      Examples of logs


9.    Statistics and monitoring
9.1.      CSV format
9.2.      Unix Socket commands

