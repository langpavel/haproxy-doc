% This is generated content
% Section 7.5.2.

\subsection{Matching contents at Layer 4 (also called Layer 6)}

\begin{verbatim}

A second set of criteria depends on data found in buffers, but which can change
during analysis. This requires that some data has been buffered, for instance
through TCP request content inspection. Please see the "tcp-request content"
keyword for more detailed information on the subject.

req_len <integer>
  Returns true when the length of the data in the request buffer matches the
  specified range. It is important to understand that this test does not
  return false as long as the buffer is changing. This means that a check with
  equality to zero will almost always immediately match at the beginning of the
  session, while a test for more data will wait for that data to come in and
  return false only when haproxy is certain that no more data will come in.
  This test was designed to be used with TCP request content inspection.

req_proto_http
  Returns true when data in the request buffer look like HTTP and correctly
  parses as such. It is the same parser as the common HTTP request parser which
  is used so there should be no surprises. This test can be used for instance
  to direct HTTP traffic to a given port and HTTPS traffic to another one
  using TCP request content inspection rules.

req_rdp_cookie       <string>
req_rdp_cookie(<name>) <string>
  Returns true when data in the request buffer look like the RDP protocol, and
  a cookie is present and equal to <string>. By default, any cookie name is
  checked, but a specific cookie name can be specified in parenthesis. The
  parser only checks for the first cookie, as illustrated in the RDP protocol
  specification. The cookie name is case insensitive. This ACL can be useful
  with the "MSTS" cookie, as it can contain the user name of the client
  connecting to the server if properly configured on the client. This can be
  used to restrict access to certain servers to certain users.

req_rdp_cookie_cnt       <integer>
req_rdp_cookie_cnt(<name>) <integer>
  Returns true when the data in the request buffer look like the RDP protocol
  and the number of RDP cookies matches the specified range (typically zero or
  one). Optionally a specific cookie name can be checked. This is a simple way
  of detecting the RDP protocol, as clients generally send the MSTS or MSTSHASH
  cookies.

req_ssl_ver <decimal>
  Returns true when data in the request buffer look like SSL, with a protocol
  version matching the specified range. Both SSLv2 hello messages and SSLv3
  messages are supported. The test tries to be strict enough to avoid being
  easily fooled. In particular, it waits for as many bytes as announced in the
  message header if this header looks valid (bound to the buffer size). Note
  that TLSv1 is announced as SSL version 3.1. This test was designed to be used
  with TCP request content inspection.

req_ssl_hello_type <integer>
  Returns true when data in the request buffer looks like a complete SSL (v3
  or superior) hello message and handshake type is equal to <integer>.
  This test was designed to be used with TCP request content inspection: an
  SSL session ID may be fetched.

rep_ssl_hello_type <integer>
  Returns true when data in the response buffer looks like a complete SSL (v3
  or superior) hello message and handshake type is equal to <integer>.
  This test was designed to be used with TCP response content inspection: a
  SSL session ID may be fetched.

wait_end
  Waits for the end of the analysis period to return true. This may be used in
  conjunction with content analysis to avoid returning a wrong verdict early.
  It may also be used to delay some actions, such as a delayed reject for some
  special addresses. Since it either stops the rules evaluation or immediately
  returns true, it is recommended to use this acl as the last one in a rule.
  Please note that the default ACL "WAIT_END" is always usable without prior
  declaration. This test was designed to be used with TCP request content
  inspection.

  Examples :
     # delay every incoming request by 2 seconds
     tcp-request inspect-delay 2s
     tcp-request content accept if WAIT_END

     # don't immediately tell bad guys they are rejected
     tcp-request inspect-delay 10s
     acl goodguys src 10.0.0.0/24
     acl badguys  src 10.0.1.0/24
     tcp-request content accept if goodguys
     tcp-request content reject if badguys WAIT_END
     tcp-request content reject


\end{verbatim}
