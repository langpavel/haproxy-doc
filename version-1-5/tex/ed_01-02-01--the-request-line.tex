% Section 1.2.1.
\subsection{The Request line}

Line 1 is the \emph{request line}. It is always composed of 3 fields:
\begin{tabbing}
\qquad \= {a \emph{HTTP method}}\qquad\= \httpmethod{GET} \\
\> {a \emph{URI}}			\> \verb|/serv/login.php?lang=en&profile=2| \\
\> {a \emph{version tag}}	\> \verb|HTTP/1.1|
\end{tabbing}

All of them are delimited by what the standard calls \emph{LWS} (linear white spaces),
which are commonly spaces, but can also be tabs or line feeds\slash carriage returns
followed by spaces\slash tabs. The method itself cannot contain any colon \CHAR{:} and
is limited to alphabetic letters. All those various combinations make it
desirable that HAProxy performs the splitting itself rather than leaving it to
the user to write a complex or inaccurate regular expression.


The URI itself can have several forms:
\begin{itemize}
\item \textbf{relative URI}

	\CHAR{/serv/login.php?lang=en\&profile=2}

	It is a complete URL without the host part. This is generally what is
	received by servers, reverse proxies and transparent proxies.

\item \textbf{absolute URI, also called a \emph{URL}}

	\CHAR{http://192.168.0.12:8080/serv/login.php?lang=en\&profile=2}

	It is composed of a \emph{scheme} (the protocol name followed by \CHAR{://}), a host
	name or address, optionally a colon \CHAR: followed by a port number, then
	a relative URI beginning at the first slash \CHAR{\slash} after the address part.
	This is generally what proxies receive, but a server supporting HTTP/1.1
	must accept this form too.

\item \textbf{star}

	\CHAR{*}

    this form is only accepted in association with the \httpmethod{OPTIONS}
    method and is not relayable. It is used to inquiry a next hop's
    capabilities.

\item \textbf{address:port combination}

    \CHAR{192.168.0.12:80}

    This is used with the \httpmethod{CONNECT} method, which is used to establish TCP
    tunnels through HTTP proxies, generally for HTTPS, but sometimes for
    other protocols too.
\end{itemize}

In a relative URI, two sub-parts are identified. The part before the question
mark is called the \emph{path}. It is typically the relative path to static objects
on the server. The part after the question mark is called the \emph{query string}.
It is mostly used with \httpmethod{GET} requests sent to dynamic scripts and is very
specific to the language, framework or application in use.



