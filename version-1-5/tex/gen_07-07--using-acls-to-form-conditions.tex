% This is generated content
% Section 7.7.

\section{Using ACLs to form conditions}

\begin{verbatim}

Some actions are only performed upon a valid condition. A condition is a
combination of ACLs with operators. 3 operators are supported :

  - AND (implicit)
  - OR  (explicit with the "or" keyword or the "||" operator)
  - Negation with the exclamation mark ("!")

A condition is formed as a disjunctive form:

   [!]acl1 [!]acl2 ... [!]acln  { or [!]acl1 [!]acl2 ... [!]acln } ...

Such conditions are generally used after an "if" or "unless" statement,
indicating when the condition will trigger the action.

For instance, to block HTTP requests to the "*" URL with methods other than
"OPTIONS", as well as POST requests without content-length, and GET or HEAD
requests with a content-length greater than 0, and finally every request which
is not either GET/HEAD/POST/OPTIONS !

   acl missing_cl hdr_cnt(Content-length) eq 0
   block if HTTP_URL_STAR !METH_OPTIONS || METH_POST missing_cl
   block if METH_GET HTTP_CONTENT
   block unless METH_GET or METH_POST or METH_OPTIONS

To select a different backend for requests to static contents on the "www" site
and to every request on the "img", "video", "download" and "ftp" hosts :

   acl url_static  path_beg         /static /images /img /css
   acl url_static  path_end         .gif .png .jpg .css .js
   acl host_www    hdr_beg(host) -i www
   acl host_static hdr_beg(host) -i img. video. download. ftp.

   # now use backend "static" for all static-only hosts, and for static urls
   # of host "www". Use backend "www" for the rest.
   use_backend static if host_static or host_www url_static
   use_backend www    if host_www

It is also possible to form rules using "anonymous ACLs". Those are unnamed ACL
expressions that are built on the fly without needing to be declared. They must
be enclosed between braces, with a space before and after each brace (because
the braces must be seen as independant words). Example :

   The following rule :

       acl missing_cl hdr_cnt(Content-length) eq 0
       block if METH_POST missing_cl

   Can also be written that way :

       block if METH_POST { hdr_cnt(Content-length) eq 0 }

It is generally not recommended to use this construct because it's a lot easier
to leave errors in the configuration when written that way. However, for very
simple rules matching only one source IP address for instance, it can make more
sense to use them than to declare ACLs with random names. Another example of
good use is the following :

   With named ACLs :

        acl site_dead nbsrv(dynamic) lt 2
        acl site_dead nbsrv(static)  lt 2
        monitor fail  if site_dead

   With anonymous ACLs :

        monitor fail if { nbsrv(dynamic) lt 2 } || { nbsrv(static) lt 2 }

See section 4.2 for detailed help on the "block" and "use_backend" keywords.


\end{verbatim}
