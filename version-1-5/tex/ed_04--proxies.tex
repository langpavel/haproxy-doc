% This is generated content
% Section 4.

\chapter{Proxies}

Proxy configuration can be located in a set of sections:
\begin{itemize}
\item \keyword{defaults} \param{name}
\item \keyword{frontend} \param{name}
\item \keyword{backend}  \param{name}
\item \keyword{listen}   \param{name}
\end{itemize}

A \keyword{defaults} section sets default parameters for all other sections following
its declaration. Those default parameters are reset by the next \keyword{defaults}
section. See below for the list of parameters which can be set in a \keyword{defaults}
section. The name is optional but its use is encouraged for better readability.


A \keyword{frontend} section describes a set of listening sockets accepting client
connections.


A \keyword{backend} section describes a set of servers to which the proxy will connect
to forward incoming connections.


A \keyword{listen} section defines a complete proxy with its frontend and backend
parts combined in one section. It is generally useful for TCP-only traffic.


All proxy names must be formed from upper and lower case letters, digits,
\CHAR{-} (dash), \CHAR{\_} (underscore) , \CHAR{.} (dot) and \CHAR{:} (colon). ACL names are
case-sensitive, which means that \CHAR{www} and \CHAR{WWW} are two different proxies.


Historically, all proxy names could overlap, it just caused troubles in the
logs. Since the introduction of content switching, it is mandatory that two
proxies with overlapping capabilities (frontend/backend) have different names.
However, it is still permitted that a frontend and a backend share the same
name, as this configuration seems to be commonly encountered.


Right now, two major proxy modes are supported: \texttt{tcp}, also known as layer 4,
and \texttt{http}, also known as layer 7. In layer 4 mode, HAProxy simply forwards
bidirectional traffic between two sides. In layer 7 mode, HAProxy analyzes the
protocol, and can interact with it by allowing, blocking, switching, adding,
modifying, or removing arbitrary contents in requests or responses, based on
arbitrary criteria.

