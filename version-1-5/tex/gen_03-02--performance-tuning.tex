% This is generated content
% Section 3.2.

\section{Performance tuning}
\begin{verbatim}
maxconn <number>
  Sets the maximum per-process number of concurrent connections to <number>. It
  is equivalent to the command-line argument "-n". Proxies will stop accepting
  connections when this limit is reached. The "ulimit-n" parameter is
  automatically adjusted according to this value. See also "ulimit-n".
\end{verbatim}

\begin{verbatim}
maxconnrate <number>
  Sets the maximum per-process number of connections per second to <number>.
  Proxies will stop accepting connections when this limit is reached. It can be
  used to limit the global capacity regardless of each frontend capacity. It is
  important to note that this can only be used as a service protection measure,
  as there will not necessarily be a fair share between frontends when the
  limit is reached, so it's a good idea to also limit each frontend to some
  value close to its expected share. Also, lowering tune.maxaccept can improve
  fairness.
\end{verbatim}

\begin{verbatim}
maxpipes <number>
  Sets the maximum per-process number of pipes to <number>. Currently, pipes
  are only used by kernel-based tcp splicing. Since a pipe contains two file
  descriptors, the "ulimit-n" value will be increased accordingly. The default
  value is maxconn/4, which seems to be more than enough for most heavy usages.
  The splice code dynamically allocates and releases pipes, and can fall back
  to standard copy, so setting this value too low may only impact performance.
\end{verbatim}

\begin{verbatim}
  Disables the use of the "epoll" event polling system on Linux. It is
  equivalent to the command-line argument "-de". The next polling system
  used will generally be "poll". See also "nosepoll", and "nopoll".
\end{verbatim}

\begin{verbatim}
  Disables the use of the "kqueue" event polling system on BSD. It is
  equivalent to the command-line argument "-dk". The next polling system
  used will generally be "poll". See also "nopoll".
\end{verbatim}

\begin{verbatim}
  Disables the use of the "poll" event polling system. It is equivalent to the
  command-line argument "-dp". The next polling system used will be "select".
  It should never be needed to disable "poll" since it's available on all
  platforms supported by HAProxy. See also "nosepoll", and "nopoll" and
  "nokqueue".
\end{verbatim}

\begin{verbatim}
  Disables the use of the "speculative epoll" event polling system on Linux. It
  is equivalent to the command-line argument "-ds". The next polling system
  used will generally be "epoll". See also "nosepoll", and "nopoll".
\end{verbatim}

\begin{verbatim}
  Disables the use of kernel tcp splicing between sockets on Linux. It is
  equivalent to the command line argument "-dS".  Data will then be copied
  using conventional and more portable recv/send calls. Kernel tcp splicing is
  limited to some very recent instances of kernel 2.6. Most versions between
  2.6.25 and 2.6.28 are buggy and will forward corrupted data, so they must not
  be used. This option makes it easier to globally disable kernel splicing in
  case of doubt. See also "option splice-auto", "option splice-request" and
  "option splice-response".
\end{verbatim}

\begin{verbatim}
spread-checks <0..50, in percent>
  Sometimes it is desirable to avoid sending health checks to servers at exact
  intervals, for instance when many logical servers are located on the same
  physical server. With the help of this parameter, it becomes possible to add
  some randomness in the check interval between 0 and +/- 50%. A value between
  2 and 5 seems to show good results. The default value remains at 0.
\end{verbatim}

\begin{verbatim}
tune.bufsize <number>
  Sets the buffer size to this size (in bytes). Lower values allow more
  sessions to coexist in the same amount of RAM, and higher values allow some
  applications with very large cookies to work. The default value is 16384 and
  can be changed at build time. It is strongly recommended not to change this
  from the default value, as very low values will break some services such as
  statistics, and values larger than default size will increase memory usage,
  possibly causing the system to run out of memory. At least the global maxconn
  parameter should be decreased by the same factor as this one is increased.
\end{verbatim}

\begin{verbatim}
tune.chksize <number>
  Sets the check buffer size to this size (in bytes). Higher values may help
  find string or regex patterns in very large pages, though doing so may imply
  more memory and CPU usage. The default value is 16384 and can be changed at
  build time. It is not recommended to change this value, but to use better
  checks whenever possible.
\end{verbatim}

\begin{verbatim}
tune.http.maxhdr <number>
  Sets the maximum number of headers in a request. When a request comes with a
  number of headers greater than this value (including the first line), it is
  rejected with a "400 Bad Request" status code. Similarly, too large responses
  are blocked with "502 Bad Gateway". The default value is 101, which is enough
  for all usages, considering that the widely deployed Apache server uses the
  same limit. It can be useful to push this limit further to temporarily allow
  a buggy application to work by the time it gets fixed. Keep in mind that each
  new header consumes 32bits of memory for each session, so don't push this
  limit too high.
\end{verbatim}

\begin{verbatim}
tune.maxaccept <number>
  Sets the maximum number of consecutive accepts that a process may perform on
  a single wake up. High values give higher priority to high connection rates,
  while lower values give higher priority to already established connections.
  This value is limited to 100 by default in single process mode. However, in
  multi-process mode (nbproc > 1), it defaults to 8 so that when one process
  wakes up, it does not take all incoming connections for itself and leaves a
  part of them to other processes. Setting this value to -1 completely disables
  the limitation. It should normally not be needed to tweak this value.
\end{verbatim}

\begin{verbatim}
tune.maxpollevents <number>
  Sets the maximum amount of events that can be processed at once in a call to
  the polling system. The default value is adapted to the operating system. It
  has been noticed that reducing it below 200 tends to slightly decrease
  latency at the expense of network bandwidth, and increasing it above 200
  tends to trade latency for slightly increased bandwidth.
\end{verbatim}

\begin{verbatim}
tune.maxrewrite <number>
  Sets the reserved buffer space to this size in bytes. The reserved space is
  used for header rewriting or appending. The first reads on sockets will never
  fill more than bufsize-maxrewrite. Historically it has defaulted to half of
  bufsize, though that does not make much sense since there are rarely large
  numbers of headers to add. Setting it too high prevents processing of large
  requests or responses. Setting it too low prevents addition of new headers
  to already large requests or to POST requests. It is generally wise to set it
  to about 1024. It is automatically readjusted to half of bufsize if it is
  larger than that. This means you don't have to worry about it when changing
  bufsize.
\end{verbatim}

\begin{verbatim}
tune.pipesize <number>
  Sets the kernel pipe buffer size to this size (in bytes). By default, pipes
  are the default size for the system. But sometimes when using TCP splicing,
  it can improve performance to increase pipe sizes, especially if it is
  suspected that pipes are not filled and that many calls to splice() are
  performed. This has an impact on the kernel's memory footprint, so this must
  not be changed if impacts are not understood.
\end{verbatim}

\begin{verbatim}
tune.rcvbuf.client <number>
tune.rcvbuf.server <number>
  Forces the kernel socket receive buffer size on the client or the server side
  to the specified value in bytes. This value applies to all TCP/HTTP frontends
  and backends. It should normally never be set, and the default size (0) lets
  the kernel autotune this value depending on the amount of available memory.
  However it can sometimes help to set it to very low values (eg: 4096) in
  order to save kernel memory by preventing it from buffering too large amounts
  of received data. Lower values will significantly increase CPU usage though.
\end{verbatim}

\begin{verbatim}
tune.sndbuf.client <number>
tune.sndbuf.server <number>
  Forces the kernel socket send buffer size on the client or the server side to
  the specified value in bytes. This value applies to all TCP/HTTP frontends
  and backends. It should normally never be set, and the default size (0) lets
  the kernel autotune this value depending on the amount of available memory.
  However it can sometimes help to set it to very low values (eg: 4096) in
  order to save kernel memory by preventing it from buffering too large amounts
  of received data. Lower values will significantly increase CPU usage though.
  Another use case is to prevent write timeouts with extremely slow clients due
  to the kernel waiting for a large part of the buffer to be read before
  notifying haproxy again.
\end{verbatim}

