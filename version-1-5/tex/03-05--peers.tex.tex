% This is generated content
% Section 3.5.

\section{Peers}

\begin{verbatim}
It is possible to synchronize server entries in stick tables between several
haproxy instances over TCP connections in a multi-master fashion. Each instance
pushes its local updates and insertions to remote peers. Server IDs are used to
identify servers remotely, so it is important that configurations look similar
or at least that the same IDs are forced on each server on all participants.
Interrupted exchanges are automatically detected and recovered from the last
known point. In addition, during a soft restart, the old process connects to
the new one using such a TCP connection to push all its entries before the new
process tries to connect to other peers. That ensures very fast replication
during a reload, it typically takes a fraction of a second even for large
tables.

peers <peersect>
  Creates a new peer list with name <peersect>. It is an independant section,
  which is referenced by one or more stick-tables.

peer <peername> <ip>:<port>
  Defines a peer inside a peers section.
  If <peername> is set to the local peer name (by default hostname, or forced
  using "-L" command line option), haproxy will listen for incoming remote peer
  connection on <ip>:<port>. Otherwise, <ip>:<port> defines where to connect to
  to join the remote peer, and <peername> is used at the protocol level to
  identify and validate the remote peer on the server side.

  During a soft restart, local peer <ip>:<port> is used by the old instance to
  connect the new one and initiate a complete replication (teaching process).

  It is strongly recommended to have the exact same peers declaration on all
  peers and to only rely on the "-L" command line argument to change the local
  peer name. This makes it easier to maintain coherent configuration files
  across all peers.

Example:
    peers mypeers
        peer haproxy1 192.168.0.1:1024
        peer haproxy2 192.168.0.2:1024
        peer haproxy3 10.2.0.1:1024

    backend mybackend
        mode tcp
        balance roundrobin
        stick-table type ip size 20k peers mypeers
        stick on src

        server srv1 192.168.0.30:80
        server srv2 192.168.0.31:80


\end{verbatim}
